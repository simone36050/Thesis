\chapter{Conclusion}
\label{sec:conclusion}

\Gls{cac} is an effective (but not particularly efficient) solution to protect sensitive data stored in the cloud from external attackers, malicious insiders, and honest-but-curious \glspl{csp}. More in detail, the computational overhead affecting \gls{cac} is --- to a significant extent --- due to the default security model usually considered in the design of \gls{cac} schemes, i.e., the \gls{csp} is always partially trusted with respect to any resource, all users are always untrusted, and all resources are always sensitive.
However, we note that such a security model may not suit every scenario, and instead the capability of expressing an alternative security model may considerably reduce the computational overhead of \gls{cac}, hence making \gls{cac} a viable solution to enforce \gls{ac} policies over cloud-hosted data. 
Therefore, in this thesis we proposed a hybrid \gls{ac} scheme comprising an extended \gls{rbac} model whose policies are automatically compiled and enforced by a centralized \gls{rbac} enforcement mechanism and \gls{cac}.
The extended \gls{rbac} model allows administrators to define security models through predicates specified for users, roles, resources, and \glspl{csp}. Besides, our extended \gls{rbac} model includes new queries allowing to fine-tune the execution of cryptographic operations in the underlying \gls{cac} scheme --- which we modified accordingly. Moreover, administrators can define predicates and redefine the entailment function in such a way to best represent their specific scenario. Also, we analyzed the computational costs of state-change rules in the extended \gls{rbac} scheme with respect to the cryptographic computations involved. Finally, we provide a proof-of-concept implementation in Prolog and a consequent demonstration of our contributions in a reasonable security model concretely defining the predicates and the entailment function used in our extended \gls{rbac} scheme, calculating the number and type of cryptographic computations involved and the corresponding execution time.

\section{Future Work}
\label{sec:conclusion:futurework}

Our work can be extended to enhance the expressiveness of the proposed \gls{rbac} model by supporting negative permissions, that is, allowing the overriding of the normal behavior of \gls{rbac} by permitting or prohibiting a specific user from accessing a particular resource. 
Additionally, we believe it is possible to reduce the number of asymmetric operations in the \gls{cac} scheme we consider in favor of symmetric operations. Moreover, to enhance efficiency, cryptographic operations may be performed only when strictly necessary. For example, the asymmetric keys of a role may be created only if a resource protected with \gls{cac} is associated with that role. 
Another possible future direction is to further investigate the computational overhead of our hybrid system and compare it with that of the vanilla \gls{cac} scheme --- as proposed in \cite{cac} --- for different sequences of state-change-rules. Additionally, it would be interesting to identify real-world scenarios over which to apply our hybrid system and evaluate the (eventual) performance improvement.
Finally, we are considering replacing Prolog with more efficient languages or technologies. For example, Datalog --- which focuses on querying data in relational databases --- is based on Prolog but offers better performance in queries with dynamic data, which is precisely the situation of our proposed hybrid scheme. 

